% Options for packages loaded elsewhere
\PassOptionsToPackage{unicode}{hyperref}
\PassOptionsToPackage{hyphens}{url}
%
\documentclass[
]{article}
\usepackage{amsmath,amssymb}
\usepackage{lmodern}
\usepackage{iftex}
\ifPDFTeX
  \usepackage[T1]{fontenc}
  \usepackage[utf8]{inputenc}
  \usepackage{textcomp} % provide euro and other symbols
\else % if luatex or xetex
  \usepackage{unicode-math}
  \defaultfontfeatures{Scale=MatchLowercase}
  \defaultfontfeatures[\rmfamily]{Ligatures=TeX,Scale=1}
\fi
% Use upquote if available, for straight quotes in verbatim environments
\IfFileExists{upquote.sty}{\usepackage{upquote}}{}
\IfFileExists{microtype.sty}{% use microtype if available
  \usepackage[]{microtype}
  \UseMicrotypeSet[protrusion]{basicmath} % disable protrusion for tt fonts
}{}
\makeatletter
\@ifundefined{KOMAClassName}{% if non-KOMA class
  \IfFileExists{parskip.sty}{%
    \usepackage{parskip}
  }{% else
    \setlength{\parindent}{0pt}
    \setlength{\parskip}{6pt plus 2pt minus 1pt}}
}{% if KOMA class
  \KOMAoptions{parskip=half}}
\makeatother
\usepackage{xcolor}
\usepackage[margin=1in]{geometry}
\usepackage{color}
\usepackage{fancyvrb}
\newcommand{\VerbBar}{|}
\newcommand{\VERB}{\Verb[commandchars=\\\{\}]}
\DefineVerbatimEnvironment{Highlighting}{Verbatim}{commandchars=\\\{\}}
% Add ',fontsize=\small' for more characters per line
\usepackage{framed}
\definecolor{shadecolor}{RGB}{248,248,248}
\newenvironment{Shaded}{\begin{snugshade}}{\end{snugshade}}
\newcommand{\AlertTok}[1]{\textcolor[rgb]{0.94,0.16,0.16}{#1}}
\newcommand{\AnnotationTok}[1]{\textcolor[rgb]{0.56,0.35,0.01}{\textbf{\textit{#1}}}}
\newcommand{\AttributeTok}[1]{\textcolor[rgb]{0.77,0.63,0.00}{#1}}
\newcommand{\BaseNTok}[1]{\textcolor[rgb]{0.00,0.00,0.81}{#1}}
\newcommand{\BuiltInTok}[1]{#1}
\newcommand{\CharTok}[1]{\textcolor[rgb]{0.31,0.60,0.02}{#1}}
\newcommand{\CommentTok}[1]{\textcolor[rgb]{0.56,0.35,0.01}{\textit{#1}}}
\newcommand{\CommentVarTok}[1]{\textcolor[rgb]{0.56,0.35,0.01}{\textbf{\textit{#1}}}}
\newcommand{\ConstantTok}[1]{\textcolor[rgb]{0.00,0.00,0.00}{#1}}
\newcommand{\ControlFlowTok}[1]{\textcolor[rgb]{0.13,0.29,0.53}{\textbf{#1}}}
\newcommand{\DataTypeTok}[1]{\textcolor[rgb]{0.13,0.29,0.53}{#1}}
\newcommand{\DecValTok}[1]{\textcolor[rgb]{0.00,0.00,0.81}{#1}}
\newcommand{\DocumentationTok}[1]{\textcolor[rgb]{0.56,0.35,0.01}{\textbf{\textit{#1}}}}
\newcommand{\ErrorTok}[1]{\textcolor[rgb]{0.64,0.00,0.00}{\textbf{#1}}}
\newcommand{\ExtensionTok}[1]{#1}
\newcommand{\FloatTok}[1]{\textcolor[rgb]{0.00,0.00,0.81}{#1}}
\newcommand{\FunctionTok}[1]{\textcolor[rgb]{0.00,0.00,0.00}{#1}}
\newcommand{\ImportTok}[1]{#1}
\newcommand{\InformationTok}[1]{\textcolor[rgb]{0.56,0.35,0.01}{\textbf{\textit{#1}}}}
\newcommand{\KeywordTok}[1]{\textcolor[rgb]{0.13,0.29,0.53}{\textbf{#1}}}
\newcommand{\NormalTok}[1]{#1}
\newcommand{\OperatorTok}[1]{\textcolor[rgb]{0.81,0.36,0.00}{\textbf{#1}}}
\newcommand{\OtherTok}[1]{\textcolor[rgb]{0.56,0.35,0.01}{#1}}
\newcommand{\PreprocessorTok}[1]{\textcolor[rgb]{0.56,0.35,0.01}{\textit{#1}}}
\newcommand{\RegionMarkerTok}[1]{#1}
\newcommand{\SpecialCharTok}[1]{\textcolor[rgb]{0.00,0.00,0.00}{#1}}
\newcommand{\SpecialStringTok}[1]{\textcolor[rgb]{0.31,0.60,0.02}{#1}}
\newcommand{\StringTok}[1]{\textcolor[rgb]{0.31,0.60,0.02}{#1}}
\newcommand{\VariableTok}[1]{\textcolor[rgb]{0.00,0.00,0.00}{#1}}
\newcommand{\VerbatimStringTok}[1]{\textcolor[rgb]{0.31,0.60,0.02}{#1}}
\newcommand{\WarningTok}[1]{\textcolor[rgb]{0.56,0.35,0.01}{\textbf{\textit{#1}}}}
\usepackage{longtable,booktabs,array}
\usepackage{calc} % for calculating minipage widths
% Correct order of tables after \paragraph or \subparagraph
\usepackage{etoolbox}
\makeatletter
\patchcmd\longtable{\par}{\if@noskipsec\mbox{}\fi\par}{}{}
\makeatother
% Allow footnotes in longtable head/foot
\IfFileExists{footnotehyper.sty}{\usepackage{footnotehyper}}{\usepackage{footnote}}
\makesavenoteenv{longtable}
\usepackage{graphicx}
\makeatletter
\def\maxwidth{\ifdim\Gin@nat@width>\linewidth\linewidth\else\Gin@nat@width\fi}
\def\maxheight{\ifdim\Gin@nat@height>\textheight\textheight\else\Gin@nat@height\fi}
\makeatother
% Scale images if necessary, so that they will not overflow the page
% margins by default, and it is still possible to overwrite the defaults
% using explicit options in \includegraphics[width, height, ...]{}
\setkeys{Gin}{width=\maxwidth,height=\maxheight,keepaspectratio}
% Set default figure placement to htbp
\makeatletter
\def\fps@figure{htbp}
\makeatother
\setlength{\emergencystretch}{3em} % prevent overfull lines
\providecommand{\tightlist}{%
  \setlength{\itemsep}{0pt}\setlength{\parskip}{0pt}}
\setcounter{secnumdepth}{-\maxdimen} % remove section numbering
\usepackage{booktabs}
\usepackage{longtable}
\usepackage{array}
\usepackage{multirow}
\usepackage{wrapfig}
\usepackage{float}
\usepackage{colortbl}
\usepackage{pdflscape}
\usepackage{tabu}
\usepackage{threeparttable}
\usepackage{threeparttablex}
\usepackage[normalem]{ulem}
\usepackage{makecell}
\usepackage{xcolor}
\ifLuaTeX
  \usepackage{selnolig}  % disable illegal ligatures
\fi
\IfFileExists{bookmark.sty}{\usepackage{bookmark}}{\usepackage{hyperref}}
\IfFileExists{xurl.sty}{\usepackage{xurl}}{} % add URL line breaks if available
\urlstyle{same} % disable monospaced font for URLs
\hypersetup{
  pdftitle={Motor Trend Regression Model},
  pdfauthor={Daniel Cevallos},
  hidelinks,
  pdfcreator={LaTeX via pandoc}}

\title{Motor Trend Regression Model}
\author{Daniel Cevallos}
\date{2023-07-11}

\begin{document}
\maketitle

\hypertarget{summary}{%
\subsection{Summary}\label{summary}}

You work for Motor Trend, a magazine about the automobile industry.
Looking at a data set of a collection of cars, they are interested in
exploring the relationship between a set of variables and miles per
gallon (MPG) (outcome). They are particularly interested in the
following two questions:

\begin{enumerate}
\def\labelenumi{\arabic{enumi}.}
\tightlist
\item
  ``Is an automatic or manual transmission better for MPG''
\item
  ``Quantify the MPG difference between automatic and manual
  transmissions''
\end{enumerate}

Some regression analysis was done, and the results obtained shows that
other than transmission type, cylinders, horsepower, and weitght are the
important factors in affecting the MPG.

\hypertarget{importing-libraries}{%
\subsection{Importing libraries}\label{importing-libraries}}

\begin{Shaded}
\begin{Highlighting}[]
\FunctionTok{library}\NormalTok{(GGally)}
\FunctionTok{library}\NormalTok{(dplyr)}
\FunctionTok{library}\NormalTok{(ggplot2)}
\FunctionTok{library}\NormalTok{(car)}
\FunctionTok{library}\NormalTok{(broom)}
\FunctionTok{library}\NormalTok{(printr)}
\FunctionTok{library}\NormalTok{(pander)}
\FunctionTok{library}\NormalTok{(kableExtra)}

\FunctionTok{theme\_set}\NormalTok{(}\FunctionTok{theme\_classic}\NormalTok{())}
\end{Highlighting}
\end{Shaded}

\hypertarget{exploratory-data-analysis}{%
\subsection{Exploratory Data Analysis}\label{exploratory-data-analysis}}

To know more about the data, you can look at the appendix section with
title ``About the data''.

\begin{Shaded}
\begin{Highlighting}[]
\FunctionTok{head}\NormalTok{(mtcars) }\SpecialCharTok{\%\textgreater{}\%}
  \FunctionTok{kbl}\NormalTok{() }\SpecialCharTok{\%\textgreater{}\%}
  \FunctionTok{kable\_styling}\NormalTok{(}\AttributeTok{bootstrap\_options =} \FunctionTok{c}\NormalTok{(}\StringTok{"striped"}\NormalTok{, }\StringTok{"hover"}\NormalTok{))}
\end{Highlighting}
\end{Shaded}

\begin{table}
\centering
\begin{tabular}[t]{l|r|r|r|r|r|r|r|r|r|r|r}
\hline
  & mpg & cyl & disp & hp & drat & wt & qsec & vs & am & gear & carb\\
\hline
Mazda RX4 & 21.0 & 6 & 160 & 110 & 3.90 & 2.620 & 16.46 & 0 & 1 & 4 & 4\\
\hline
Mazda RX4 Wag & 21.0 & 6 & 160 & 110 & 3.90 & 2.875 & 17.02 & 0 & 1 & 4 & 4\\
\hline
Datsun 710 & 22.8 & 4 & 108 & 93 & 3.85 & 2.320 & 18.61 & 1 & 1 & 4 & 1\\
\hline
Hornet 4 Drive & 21.4 & 6 & 258 & 110 & 3.08 & 3.215 & 19.44 & 1 & 0 & 3 & 1\\
\hline
Hornet Sportabout & 18.7 & 8 & 360 & 175 & 3.15 & 3.440 & 17.02 & 0 & 0 & 3 & 2\\
\hline
Valiant & 18.1 & 6 & 225 & 105 & 2.76 & 3.460 & 20.22 & 1 & 0 & 3 & 1\\
\hline
\end{tabular}
\end{table}

Since \texttt{vs} and \texttt{am} are factor variables, we'll be
factorizing them to get more interpretable outputs in regression.

\begin{Shaded}
\begin{Highlighting}[]
\CommentTok{\# factoring categorical variables for regression}
\NormalTok{mtcars }\OtherTok{\textless{}{-}}\NormalTok{ mtcars }\SpecialCharTok{\%\textgreater{}\%}
    \FunctionTok{mutate}\NormalTok{(}\AttributeTok{am =} \FunctionTok{factor}\NormalTok{(am, }\AttributeTok{labels =} \FunctionTok{c}\NormalTok{(}\StringTok{"automatic"}\NormalTok{, }\StringTok{"manual"}\NormalTok{))) }\SpecialCharTok{\%\textgreater{}\%}
    \FunctionTok{mutate}\NormalTok{(}\AttributeTok{vs =} \FunctionTok{factor}\NormalTok{(vs))}
\end{Highlighting}
\end{Shaded}

The question focuses on the am variable, which is transmission type -
automatic or manual. To answer the question, we can plot a box plot to
see the difference between automatic and manual.

Based on the box plot in the appendix, we can form a hypothesis that the
manual cars have higher miles per gallon, which means it has higher fuel
efficiency as compared to automatic cars. o test for this claim, we can
use a statistical test such as the t test.

\hypertarget{two-samples-t-test}{%
\subsubsection{Two samples t test}\label{two-samples-t-test}}

\begin{Shaded}
\begin{Highlighting}[]
\FunctionTok{panderOptions}\NormalTok{(}\StringTok{\textquotesingle{}table.split.table\textquotesingle{}}\NormalTok{, }\StringTok{\textquotesingle{}50\textquotesingle{}}\NormalTok{)}
\FunctionTok{pander}\NormalTok{(}\FunctionTok{t.test}\NormalTok{(mtcars}\SpecialCharTok{$}\NormalTok{mpg }\SpecialCharTok{\textasciitilde{}}\NormalTok{ mtcars}\SpecialCharTok{$}\NormalTok{am))}
\end{Highlighting}
\end{Shaded}

\begin{longtable}[]{@{}
  >{\centering\arraybackslash}p{(\columnwidth - 10\tabcolsep) * \real{0.1491}}
  >{\centering\arraybackslash}p{(\columnwidth - 10\tabcolsep) * \real{0.0702}}
  >{\centering\arraybackslash}p{(\columnwidth - 10\tabcolsep) * \real{0.1316}}
  >{\centering\arraybackslash}p{(\columnwidth - 10\tabcolsep) * \real{0.2193}}
  >{\centering\arraybackslash}p{(\columnwidth - 10\tabcolsep) * \real{0.2281}}
  >{\centering\arraybackslash}p{(\columnwidth - 10\tabcolsep) * \real{0.2018}}@{}}
\caption{Welch Two Sample t-test: \texttt{mtcars\$mpg} by
\texttt{mtcars\$am}}\tabularnewline
\toprule()
\begin{minipage}[b]{\linewidth}\centering
Test statistic
\end{minipage} & \begin{minipage}[b]{\linewidth}\centering
df
\end{minipage} & \begin{minipage}[b]{\linewidth}\centering
P value
\end{minipage} & \begin{minipage}[b]{\linewidth}\centering
Alternative hypothesis
\end{minipage} & \begin{minipage}[b]{\linewidth}\centering
mean in group automatic
\end{minipage} & \begin{minipage}[b]{\linewidth}\centering
mean in group manual
\end{minipage} \\
\midrule()
\endfirsthead
\toprule()
\begin{minipage}[b]{\linewidth}\centering
Test statistic
\end{minipage} & \begin{minipage}[b]{\linewidth}\centering
df
\end{minipage} & \begin{minipage}[b]{\linewidth}\centering
P value
\end{minipage} & \begin{minipage}[b]{\linewidth}\centering
Alternative hypothesis
\end{minipage} & \begin{minipage}[b]{\linewidth}\centering
mean in group automatic
\end{minipage} & \begin{minipage}[b]{\linewidth}\centering
mean in group manual
\end{minipage} \\
\midrule()
\endhead
-3.767 & 18.33 & 0.001374 * * & two.sided & 17.15 & 24.39 \\
\bottomrule()
\end{longtable}

From the t test, we get a significant p-value, this means we can reject
the null that there is no difference between auto and manual cars. In
other words, the probability that the difference in these two groups
appeared by chance is very low. Observing the confidence interval, we
are 95\% confident that the true difference between automatic and manual
cars are between 3.2 and 11.3.

Since we'll be fitting regression models on this data, it's useful to
look pairwise scatter plots as this gives us a quick look into the
relationship between variables. This plot can be observed at the
appendix.

\hypertarget{regression-model-and-hypothesis-testing}{%
\subsection{Regression Model and Hypothesis
testing}\label{regression-model-and-hypothesis-testing}}

\hypertarget{simple-linear-regression-model}{%
\subsubsection{Simple Linear Regression
Model}\label{simple-linear-regression-model}}

Since Motor Trends is more interested in the am variable, we'll be
fitting it to the model and observe the results.

\begin{Shaded}
\begin{Highlighting}[]
\NormalTok{fit\_am }\OtherTok{\textless{}{-}} \FunctionTok{lm}\NormalTok{(mpg }\SpecialCharTok{\textasciitilde{}}\NormalTok{ am, mtcars)}
\FunctionTok{summary}\NormalTok{(fit\_am) }
\end{Highlighting}
\end{Shaded}

\begin{verbatim}
## 
## Call:
## lm(formula = mpg ~ am, data = mtcars)
## 
## Residuals:
##     Min      1Q  Median      3Q     Max 
## -9.3923 -3.0923 -0.2974  3.2439  9.5077 
## 
## Coefficients:
##             Estimate Std. Error t value Pr(>|t|)    
## (Intercept)   17.147      1.125  15.247 1.13e-15 ***
## ammanual       7.245      1.764   4.106 0.000285 ***
## ---
## Signif. codes:  0 '***' 0.001 '**' 0.01 '*' 0.05 '.' 0.1 ' ' 1
## 
## Residual standard error: 4.902 on 30 degrees of freedom
## Multiple R-squared:  0.3598, Adjusted R-squared:  0.3385 
## F-statistic: 16.86 on 1 and 30 DF,  p-value: 0.000285
\end{verbatim}

The reference variable follows an alphabetical order, so interpreting
the coefficients, note that the reference variable in this case is
automatic transmission.

\begin{itemize}
\tightlist
\item
  The \texttt{intercept} here shows us that 17.15 is mean mpg for
  automatic transmission.
\item
  The \texttt{slope} coefficient shows us that 7.24 is the change in
  mean between the automatic and manual transmission (this can be
  observed from the box plot previously)
\item
  The \texttt{p-value} for the slope coefficient tells us that the mean
  difference between auto and manual transmission is significant, and
  thus we can conclude that manual transmission is more fuel efficient
  as compared to automatic.
\item
  The r squared for our model is low, with only 36\% of variation
  explained by the model. This makes sense because models with only one
  variable usually isn't enough.
\end{itemize}

Simple linear regression is usually insufficient in terms of creating a
good model that can predict mpg because there are other predictor
variables or regressors that can help explain more variation in the
model. Thus, this is where multivariate linear regression can help us
fit more variables to produce a better model.

\hypertarget{multivariate-linear-regression-model}{%
\subsubsection{Multivariate Linear Regression
Model}\label{multivariate-linear-regression-model}}

The goal is to create a model that best predicts mpg, or ultimately fuel
efficiency. This means that each of the predictors variables should have
a statistically significant p-value and are not correlated in any way
(this will be tested with the Variance Inflation factor later on). Model
fit can also be tested with anova, where you can observe whether adding
a variable explains away a significant portion of variation (or looking
at the p-value).

The challenge with Multivariate regression is which variables you should
include or remove. Here we see what happens if we include all the
variables in the data.

\begin{Shaded}
\begin{Highlighting}[]
\NormalTok{full.model }\OtherTok{\textless{}{-}} \FunctionTok{lm}\NormalTok{(mpg }\SpecialCharTok{\textasciitilde{}}\NormalTok{ ., }\AttributeTok{data =}\NormalTok{ mtcars)}
\FunctionTok{summary}\NormalTok{(full.model) }
\end{Highlighting}
\end{Shaded}

\begin{verbatim}
## 
## Call:
## lm(formula = mpg ~ ., data = mtcars)
## 
## Residuals:
##     Min      1Q  Median      3Q     Max 
## -3.4506 -1.6044 -0.1196  1.2193  4.6271 
## 
## Coefficients:
##             Estimate Std. Error t value Pr(>|t|)  
## (Intercept) 12.30337   18.71788   0.657   0.5181  
## cyl         -0.11144    1.04502  -0.107   0.9161  
## disp         0.01334    0.01786   0.747   0.4635  
## hp          -0.02148    0.02177  -0.987   0.3350  
## drat         0.78711    1.63537   0.481   0.6353  
## wt          -3.71530    1.89441  -1.961   0.0633 .
## qsec         0.82104    0.73084   1.123   0.2739  
## vs1          0.31776    2.10451   0.151   0.8814  
## ammanual     2.52023    2.05665   1.225   0.2340  
## gear         0.65541    1.49326   0.439   0.6652  
## carb        -0.19942    0.82875  -0.241   0.8122  
## ---
## Signif. codes:  0 '***' 0.001 '**' 0.01 '*' 0.05 '.' 0.1 ' ' 1
## 
## Residual standard error: 2.65 on 21 degrees of freedom
## Multiple R-squared:  0.869,  Adjusted R-squared:  0.8066 
## F-statistic: 13.93 on 10 and 21 DF,  p-value: 3.793e-07
\end{verbatim}

You can see that almost all (besides wt) the variables have p-values
that are not significant

An issue with multivariate regression is certain variables may be
correlated with each other, which can increase the standard error of
other variables. To assess colinearity, we can use the Variance
inflation factor, which r has a nifty function (vif) that does so.

\begin{Shaded}
\begin{Highlighting}[]
\FunctionTok{rbind}\NormalTok{(}\FunctionTok{vif}\NormalTok{(full.model))  }\SpecialCharTok{\%\textgreater{}\%}
  \FunctionTok{kbl}\NormalTok{() }\SpecialCharTok{\%\textgreater{}\%}
  \FunctionTok{kable\_styling}\NormalTok{(}\AttributeTok{bootstrap\_options =} \FunctionTok{c}\NormalTok{(}\StringTok{"striped"}\NormalTok{, }\StringTok{"hover"}\NormalTok{))}
\end{Highlighting}
\end{Shaded}

\begin{table}
\centering
\begin{tabular}[t]{r|r|r|r|r|r|r|r|r|r}
\hline
cyl & disp & hp & drat & wt & qsec & vs & am & gear & carb\\
\hline
15.37383 & 21.62024 & 9.832037 & 3.37462 & 15.16489 & 7.527958 & 4.965873 & 4.648487 & 5.357452 & 7.908747\\
\hline
\end{tabular}
\end{table}

We see some of the variables have really high VIF (more than 10) which
shows signs of colinearity.

\hypertarget{stepwise-regression-model}{%
\subsubsection{Stepwise regression
model}\label{stepwise-regression-model}}

There are many ways to test for different variables to choose the best
model, here I will be using the stepwise selection method to help find
the predictor variables that can best explain MPG.

\begin{Shaded}
\begin{Highlighting}[]
\NormalTok{bestModel }\OtherTok{\textless{}{-}} \FunctionTok{step}\NormalTok{(full.model, }\AttributeTok{direction =} \StringTok{"both"}\NormalTok{,}
                  \AttributeTok{trace =} \ConstantTok{FALSE}\NormalTok{)}
\FunctionTok{summary}\NormalTok{(bestModel)}
\end{Highlighting}
\end{Shaded}

\begin{verbatim}
## 
## Call:
## lm(formula = mpg ~ wt + qsec + am, data = mtcars)
## 
## Residuals:
##     Min      1Q  Median      3Q     Max 
## -3.4811 -1.5555 -0.7257  1.4110  4.6610 
## 
## Coefficients:
##             Estimate Std. Error t value Pr(>|t|)    
## (Intercept)   9.6178     6.9596   1.382 0.177915    
## wt           -3.9165     0.7112  -5.507 6.95e-06 ***
## qsec          1.2259     0.2887   4.247 0.000216 ***
## ammanual      2.9358     1.4109   2.081 0.046716 *  
## ---
## Signif. codes:  0 '***' 0.001 '**' 0.01 '*' 0.05 '.' 0.1 ' ' 1
## 
## Residual standard error: 2.459 on 28 degrees of freedom
## Multiple R-squared:  0.8497, Adjusted R-squared:  0.8336 
## F-statistic: 52.75 on 3 and 28 DF,  p-value: 1.21e-11
\end{verbatim}

Using the stepwise method, we end up with 3 predictor variables,
\texttt{wt}, \texttt{qsec} and \texttt{am}.

\begin{itemize}
\tightlist
\item
  all three variables have significant p-values, which suggest that they
  are all important addition to the model for predicting mpg.
\item
  note that our am variable has a less significant p-value after
  adjusting for variabes \texttt{wt} and \texttt{qsec}
\item
  after adjusting for other predictor variables, our coefficient for am
  went down to 2.94, and our pvalue became less significant.
\item
  The r squared value denotes how much of the variation in mpg is
  explained. Our best model explains around 84\% of the variation, which
  indicates it's a good model.
\end{itemize}

\hypertarget{regression-diagnostics}{%
\subsection{Regression diagnostics}\label{regression-diagnostics}}

\begin{Shaded}
\begin{Highlighting}[]
\NormalTok{vif }\OtherTok{\textless{}{-}} \FunctionTok{cbind}\NormalTok{(}\FunctionTok{vif}\NormalTok{(bestModel))}
\FunctionTok{colnames}\NormalTok{(vif) }\OtherTok{\textless{}{-}} \StringTok{"VIF of bestmodel"}
\NormalTok{vif  }\SpecialCharTok{\%\textgreater{}\%}
  \FunctionTok{kbl}\NormalTok{() }\SpecialCharTok{\%\textgreater{}\%}
  \FunctionTok{kable\_styling}\NormalTok{(}\AttributeTok{bootstrap\_options =} \FunctionTok{c}\NormalTok{(}\StringTok{"striped"}\NormalTok{, }\StringTok{"hover"}\NormalTok{), }\AttributeTok{full\_width =}\NormalTok{ F)}
\end{Highlighting}
\end{Shaded}

\begin{table}
\centering
\begin{tabular}[t]{l|r}
\hline
  & VIF of bestmodel\\
\hline
wt & 2.482951\\
\hline
qsec & 1.364339\\
\hline
am & 2.541437\\
\hline
\end{tabular}
\end{table}

The variance inflation factor of all three of our variables are small,
which means they are not highly correlated.

\hypertarget{anova-test-on-nested-models}{%
\subsubsection{Anova test on nested
models}\label{anova-test-on-nested-models}}

Anova is a useful statistical tool to use on nested models. With it, we
can interpret what the effects of adding a new variable are on the
coefficients and the p-values

\begin{Shaded}
\begin{Highlighting}[]
\NormalTok{fit0 }\OtherTok{\textless{}{-}} \FunctionTok{lm}\NormalTok{(mpg }\SpecialCharTok{\textasciitilde{}}\NormalTok{ am, mtcars)}
\NormalTok{fit1 }\OtherTok{\textless{}{-}} \FunctionTok{update}\NormalTok{(fit0, mpg }\SpecialCharTok{\textasciitilde{}}\NormalTok{ am }\SpecialCharTok{+}\NormalTok{ wt)}
\NormalTok{fit2 }\OtherTok{\textless{}{-}} \FunctionTok{update}\NormalTok{(fit1, mpg }\SpecialCharTok{\textasciitilde{}}\NormalTok{ am }\SpecialCharTok{+}\NormalTok{ wt }\SpecialCharTok{+}\NormalTok{ qsec)}
\NormalTok{fit3 }\OtherTok{\textless{}{-}} \FunctionTok{update}\NormalTok{(fit2, mpg }\SpecialCharTok{\textasciitilde{}}\NormalTok{ am }\SpecialCharTok{+}\NormalTok{ wt }\SpecialCharTok{+}\NormalTok{ qsec }\SpecialCharTok{+}\NormalTok{ disp)}
\NormalTok{fit4 }\OtherTok{\textless{}{-}} \FunctionTok{update}\NormalTok{(fit3, mpg }\SpecialCharTok{\textasciitilde{}}\NormalTok{ am }\SpecialCharTok{+}\NormalTok{ wt }\SpecialCharTok{+}\NormalTok{ qsec }\SpecialCharTok{+}\NormalTok{ disp }\SpecialCharTok{+}\NormalTok{ hp)}

\FunctionTok{anova}\NormalTok{(fit0, fit1, fit2, fit3, fit4)  }\SpecialCharTok{\%\textgreater{}\%}
  \FunctionTok{kbl}\NormalTok{() }\SpecialCharTok{\%\textgreater{}\%}
  \FunctionTok{kable\_styling}\NormalTok{(}\AttributeTok{bootstrap\_options =} \FunctionTok{c}\NormalTok{(}\StringTok{"striped"}\NormalTok{, }\StringTok{"hover"}\NormalTok{))}
\end{Highlighting}
\end{Shaded}

\begin{table}
\centering
\begin{tabular}[t]{r|r|r|r|r|r}
\hline
Res.Df & RSS & Df & Sum of Sq & F & Pr(>F)\\
\hline
30 & 720.8966 & NA & NA & NA & NA\\
\hline
29 & 278.3197 & 1 & 442.57690 & 74.9945513 & 0.0000000\\
\hline
28 & 169.2859 & 1 & 109.03377 & 18.4757461 & 0.0002140\\
\hline
27 & 166.0099 & 1 & 3.27607 & 0.5551293 & 0.4629119\\
\hline
26 & 153.4378 & 1 & 12.57205 & 2.1303314 & 0.1563873\\
\hline
\end{tabular}
\end{table}

Looking at the results, we see how our best fit gives us a significant
result (consistent with our stepwise selection model), but adding the
variables \texttt{disp} and \texttt{hp} gives us p-values that are not
significant, thus a worse model.

\hypertarget{residual-diganostic-plots}{%
\subsubsection{Residual diganostic
plots}\label{residual-diganostic-plots}}

To diagnose a regression model it's also important to look at the
residual diagnostics, which can be seen at the appendix.

\begin{itemize}
\tightlist
\item
  From our residual vs fitted plot, we don't see any distinct patterns,
  which is a good sign
\item
  Our normal Q-Q plot shows that our standardized residuals are
  considerably normal, and doesn't deviate that much from the line.
\item
  scale-location is compares standardized residuals with fitted values,
  and we don't see any patterns as well
\item
  Our residual vs leverage plot don't contain any systematic patterns.
\end{itemize}

\pagebreak

\hypertarget{appendix}{%
\subsection{Appendix}\label{appendix}}

\hypertarget{code}{%
\subsubsection{Code}\label{code}}

\begin{Shaded}
\begin{Highlighting}[]
\FunctionTok{library}\NormalTok{(GGally)}
\FunctionTok{library}\NormalTok{(dplyr)}
\FunctionTok{library}\NormalTok{(ggplot2)}
\FunctionTok{library}\NormalTok{(car)}
\FunctionTok{library}\NormalTok{(broom)}
\FunctionTok{library}\NormalTok{(printr)}
\FunctionTok{library}\NormalTok{(pander)}
\FunctionTok{library}\NormalTok{(kableExtra)}

\FunctionTok{theme\_set}\NormalTok{(}\FunctionTok{theme\_classic}\NormalTok{())}
\end{Highlighting}
\end{Shaded}

\begin{Shaded}
\begin{Highlighting}[]
\FunctionTok{head}\NormalTok{(mtcars) }\SpecialCharTok{\%\textgreater{}\%}
  \FunctionTok{kbl}\NormalTok{() }\SpecialCharTok{\%\textgreater{}\%}
  \FunctionTok{kable\_styling}\NormalTok{(}\AttributeTok{bootstrap\_options =} \FunctionTok{c}\NormalTok{(}\StringTok{"striped"}\NormalTok{, }\StringTok{"hover"}\NormalTok{))}
\end{Highlighting}
\end{Shaded}

\begin{Shaded}
\begin{Highlighting}[]
\CommentTok{\# factoring categorical variables for regression}
\NormalTok{mtcars }\OtherTok{\textless{}{-}}\NormalTok{ mtcars }\SpecialCharTok{\%\textgreater{}\%}
    \FunctionTok{mutate}\NormalTok{(}\AttributeTok{am =} \FunctionTok{factor}\NormalTok{(am, }\AttributeTok{labels =} \FunctionTok{c}\NormalTok{(}\StringTok{"automatic"}\NormalTok{, }\StringTok{"manual"}\NormalTok{))) }\SpecialCharTok{\%\textgreater{}\%}
    \FunctionTok{mutate}\NormalTok{(}\AttributeTok{vs =} \FunctionTok{factor}\NormalTok{(vs))}
\end{Highlighting}
\end{Shaded}

\begin{Shaded}
\begin{Highlighting}[]
\FunctionTok{panderOptions}\NormalTok{(}\StringTok{\textquotesingle{}table.split.table\textquotesingle{}}\NormalTok{, }\StringTok{\textquotesingle{}50\textquotesingle{}}\NormalTok{)}
\FunctionTok{pander}\NormalTok{(}\FunctionTok{t.test}\NormalTok{(mtcars}\SpecialCharTok{$}\NormalTok{mpg }\SpecialCharTok{\textasciitilde{}}\NormalTok{ mtcars}\SpecialCharTok{$}\NormalTok{am))}
\end{Highlighting}
\end{Shaded}

\begin{Shaded}
\begin{Highlighting}[]
\NormalTok{fit\_am }\OtherTok{\textless{}{-}} \FunctionTok{lm}\NormalTok{(mpg }\SpecialCharTok{\textasciitilde{}}\NormalTok{ am, mtcars)}
\FunctionTok{summary}\NormalTok{(fit\_am) }
\end{Highlighting}
\end{Shaded}

\begin{Shaded}
\begin{Highlighting}[]
\NormalTok{full.model }\OtherTok{\textless{}{-}} \FunctionTok{lm}\NormalTok{(mpg }\SpecialCharTok{\textasciitilde{}}\NormalTok{ ., }\AttributeTok{data =}\NormalTok{ mtcars)}
\FunctionTok{summary}\NormalTok{(full.model) }
\end{Highlighting}
\end{Shaded}

\begin{Shaded}
\begin{Highlighting}[]
\FunctionTok{rbind}\NormalTok{(}\FunctionTok{vif}\NormalTok{(full.model))  }\SpecialCharTok{\%\textgreater{}\%}
  \FunctionTok{kbl}\NormalTok{() }\SpecialCharTok{\%\textgreater{}\%}
  \FunctionTok{kable\_styling}\NormalTok{(}\AttributeTok{bootstrap\_options =} \FunctionTok{c}\NormalTok{(}\StringTok{"striped"}\NormalTok{, }\StringTok{"hover"}\NormalTok{))}
\end{Highlighting}
\end{Shaded}

\begin{Shaded}
\begin{Highlighting}[]
\NormalTok{bestModel }\OtherTok{\textless{}{-}} \FunctionTok{step}\NormalTok{(full.model, }\AttributeTok{direction =} \StringTok{"both"}\NormalTok{,}
                  \AttributeTok{trace =} \ConstantTok{FALSE}\NormalTok{)}
\FunctionTok{summary}\NormalTok{(bestModel)}
\end{Highlighting}
\end{Shaded}

\begin{Shaded}
\begin{Highlighting}[]
\NormalTok{vif }\OtherTok{\textless{}{-}} \FunctionTok{cbind}\NormalTok{(}\FunctionTok{vif}\NormalTok{(bestModel))}
\FunctionTok{colnames}\NormalTok{(vif) }\OtherTok{\textless{}{-}} \StringTok{"VIF of bestmodel"}
\NormalTok{vif  }\SpecialCharTok{\%\textgreater{}\%}
  \FunctionTok{kbl}\NormalTok{() }\SpecialCharTok{\%\textgreater{}\%}
  \FunctionTok{kable\_styling}\NormalTok{(}\AttributeTok{bootstrap\_options =} \FunctionTok{c}\NormalTok{(}\StringTok{"striped"}\NormalTok{, }\StringTok{"hover"}\NormalTok{), }\AttributeTok{full\_width =}\NormalTok{ F)}
\end{Highlighting}
\end{Shaded}

\begin{Shaded}
\begin{Highlighting}[]
\NormalTok{fit0 }\OtherTok{\textless{}{-}} \FunctionTok{lm}\NormalTok{(mpg }\SpecialCharTok{\textasciitilde{}}\NormalTok{ am, mtcars)}
\NormalTok{fit1 }\OtherTok{\textless{}{-}} \FunctionTok{update}\NormalTok{(fit0, mpg }\SpecialCharTok{\textasciitilde{}}\NormalTok{ am }\SpecialCharTok{+}\NormalTok{ wt)}
\NormalTok{fit2 }\OtherTok{\textless{}{-}} \FunctionTok{update}\NormalTok{(fit1, mpg }\SpecialCharTok{\textasciitilde{}}\NormalTok{ am }\SpecialCharTok{+}\NormalTok{ wt }\SpecialCharTok{+}\NormalTok{ qsec)}
\NormalTok{fit3 }\OtherTok{\textless{}{-}} \FunctionTok{update}\NormalTok{(fit2, mpg }\SpecialCharTok{\textasciitilde{}}\NormalTok{ am }\SpecialCharTok{+}\NormalTok{ wt }\SpecialCharTok{+}\NormalTok{ qsec }\SpecialCharTok{+}\NormalTok{ disp)}
\NormalTok{fit4 }\OtherTok{\textless{}{-}} \FunctionTok{update}\NormalTok{(fit3, mpg }\SpecialCharTok{\textasciitilde{}}\NormalTok{ am }\SpecialCharTok{+}\NormalTok{ wt }\SpecialCharTok{+}\NormalTok{ qsec }\SpecialCharTok{+}\NormalTok{ disp }\SpecialCharTok{+}\NormalTok{ hp)}

\FunctionTok{anova}\NormalTok{(fit0, fit1, fit2, fit3, fit4)  }\SpecialCharTok{\%\textgreater{}\%}
  \FunctionTok{kbl}\NormalTok{() }\SpecialCharTok{\%\textgreater{}\%}
  \FunctionTok{kable\_styling}\NormalTok{(}\AttributeTok{bootstrap\_options =} \FunctionTok{c}\NormalTok{(}\StringTok{"striped"}\NormalTok{, }\StringTok{"hover"}\NormalTok{))}
\end{Highlighting}
\end{Shaded}

\pagebreak

\hypertarget{about-the-data}{%
\subsubsection{About the data}\label{about-the-data}}

\begin{verbatim}
A data frame with 32 observations on 11 (numeric) variables.

[, 1]   mpg     Miles/(US) gallon
[, 2]   cyl     Number of cylinders
[, 3]   disp    Displacement (cu.in.)
[, 4]   hp      Gross horsepower
[, 5]   drat    Rear axle ratio
[, 6]   wt      Weight (1000 lbs)
[, 7]   qsec    1/4 mile time
[, 8]   vs      Engine (0 = V-shaped, 1 = straight)
[, 9]   am      Transmission (0 = automatic, 1 = manual)
[,10]   gear    Number of forward gears
[,11]   carb    Number of carburetors
\end{verbatim}

\hypertarget{box-plot}{%
\subsubsection{Box plot}\label{box-plot}}

\begin{Shaded}
\begin{Highlighting}[]
\FunctionTok{ggplot}\NormalTok{(mtcars, }\FunctionTok{aes}\NormalTok{(}\FunctionTok{factor}\NormalTok{(am, }\AttributeTok{labels =} \FunctionTok{c}\NormalTok{(}
    \StringTok{"automatic"}\NormalTok{, }\StringTok{"manual"}
\NormalTok{)), mpg, }\AttributeTok{fill =} \FunctionTok{factor}\NormalTok{(am))) }\SpecialCharTok{+}
    \FunctionTok{geom\_boxplot}\NormalTok{() }\SpecialCharTok{+}
    \FunctionTok{labs}\NormalTok{(}\AttributeTok{x =} \StringTok{"Transmission type"}\NormalTok{, }\AttributeTok{y=}\StringTok{"Miles per gallon"}\NormalTok{)}
\end{Highlighting}
\end{Shaded}

\includegraphics{Figs/boxplot-1.pdf}

\pagebreak

\hypertarget{pairs-plot}{%
\subsubsection{Pairs plot}\label{pairs-plot}}

\includegraphics{Figs/ggpair-1.pdf}

\hypertarget{regression-diagnostics-plot}{%
\subsubsection{Regression diagnostics
plot}\label{regression-diagnostics-plot}}

\begin{Shaded}
\begin{Highlighting}[]
\FunctionTok{par}\NormalTok{(}\AttributeTok{mfrow =} \FunctionTok{c}\NormalTok{(}\DecValTok{2}\NormalTok{, }\DecValTok{2}\NormalTok{))}
\FunctionTok{plot}\NormalTok{(bestModel)}
\end{Highlighting}
\end{Shaded}

\includegraphics{Figs/reg_diag_plot-1.pdf}

\hypertarget{conclusion}{%
\subsubsection{Conclusion}\label{conclusion}}

On average, manual transmission is better than automatic transmission by
1.81mpg. However, transmission type is not the only factor accounting
for MPG, cylinders, horsepower, and weitght are the important factors in
affecting the MPG.

\end{document}
